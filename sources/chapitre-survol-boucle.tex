% === Cours de Java
% === Chapitre : Survol 

\section{Les boucles (survol)}

\leconwithabstract{Nous voyons comment traduire les algorithmes contenant des boucles que vous �crivez au cours de Logique}

\begin{frame}[fragile]{Instructions r�p�titives}
Le \emph{Tant que} : 
\begin{Java}
  while ( condition ) {
    instructions
  }
\end{Java} 
\bigskip
\emph{Exemple} :
\begin{Java}
  int puissance = 1;
  while ( puissance < 1000 ) {
    System.out.println(puissance);
    puissance = 2 * puissance;
  }
\end{Java}
\end{frame}

\begin{frame}[fragile]{Exemple} 
\begin{Java}
import java.util.Scanner;
public class Exemple {
  /**
   * Affiche la somme d'entiers positifs entr�s au clavier.
   * S'arr�te d�s qu'une valeur nulle ou n�gative est donn�e.
   * @param args non utilis�
   */
  public static void main(String[] args) {
      Scanner clavier = new Scanner(System.in);
      int nb;
      int somme = 0;
      nb = clavier.nextInt();
      while ( nb > 0 ) {
         somme = somme + nb;
         nb = clavier.nextInt();
      }
      System.out.println(somme);
  }
}
\end{Java}
\end{frame}

\begin{frame}[fragile]{Instructions r�p�titives}
Le \emph{Pour} : 
\begin{Java}
  for ( int i=d�but; i<=fin; i=i+pas ) {
    instructions
  }
\end{Java} 
\bigskip
\emph{Exemple} :
\begin{Java}
  for ( int i=1; i<=10; i=i+1 ) {
    System.out.println(i);
  }
\end{Java}
\end{frame}

\begin{frame}[fragile]{Exemple} 
\begin{Java}
public class Exemple {
  /**
   * Affiche la somme des nombres pairs entre 2 et 100.
   * @param args non utilis�
   */
  public static void main(String[] args) {
      int somme;

      somme = 0;
      for ( int i=2; i<=100; i=i+2 ) {
         somme = somme + i;
      }
      System.out.println(somme);
  }
}
\end{Java}
\end{frame}

\begin{frame}[fragile]{Exemple} 
\begin{Java}
public class Exemple {
  /**
   * Affiche un compte � rebours � partir de 10.
   * @param args non utilis�
   */
  public static void main(String[] args) {

      for ( int i=10; i>=1; i=i-1 ) {
         System.out.println(i);
      }
      System.out.println("Partez !");
  }
}
\end{Java}
\end{frame}

\begin{frame}[fragile]{Instructions r�p�titives}
\java|i++| est un raccourci pour \java|i=i+1|
\\\bigskip
\emph{Exemple} :
\begin{Java}
  for ( int i=1; i<=n; i++ ) {
    System.out.println(i);
  }
\end{Java} 
\end{frame}

\begin{frame}[fragile]{�tude de cas} 
\emph{Objectif} : lecture d'une donn�e enti�re positive
\begin{itemize}
\item \emph{�tape 1} : lire un entier
\end{itemize}
\begin{Java}
/**
  * Lit un entier au clavier.
  * Les valeurs non enti�res sont pass�es.
  * @return l'entier lu.
  */
public static int lireEntier() {
    Scanner clavier = new Scanner(System.in);
    int nb;
    // Tant que ce n'est pas un entier au clavier
    while ( !clavier.hasNextInt() ) {
        clavier.next(); // le lire, le passer
    }
    nb = clavier.nextInt();
    return nb;
}
\end{Java}
\end{frame}

\begin{frame}[fragile]{�tude de cas} 
\begin{itemize}
\item \emph{�tape 2} : lire un entier positif
\end{itemize}
\begin{Java}
/**
  * Lit un entier au clavier.
  * Les valeurs non enti�res, nulles ou n�gatives sont pass�es.
  * @return l'entier lu.
  */
public static int lirePositif() {
    int nb;
    nb = lireEntier();
    while (nb<=0) {
      nb = lireEntier();
    }
    return nb;
}
\end{Java}
\end{frame}



